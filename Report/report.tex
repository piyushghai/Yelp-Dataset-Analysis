%%%%%%%%%%%%%%%%%%%%%%%%%%%%%%%%%%%%%%%%%
% Short Sectioned Assignment
% LaTeX Template
% Version 1.0 (5/5/12)
%
% This template has been downloaded from:
% http://www.LaTeXTemplates.com
%
% Original author:
% Frits Wenneker (http://www.howtotex.com)
%
% License:
% CC BY-NC-SA 3.0 (http://creativecommons.org/licenses/by-nc-sa/3.0/)
%
%%%%%%%%%%%%%%%%%%%%%%%%%%%%%%%%%%%%%%%%%

%----------------------------------------------------------------------------------------
%	PACKAGES AND OTHER DOCUMENT CONFIGURATIONS
%----------------------------------------------------------------------------------------

\documentclass[paper=a4, fontsize=11pt]{scrartcl} % A4 paper and 10pt font size

\usepackage[T1]{fontenc} % Use 8-bit encoding that has 256 glyphs
%\usepackage{fourier} % Use the Adobe Utopia font for the document - comment this line to return to the LaTeX default
\usepackage[english]{babel} % English language/hyphenation
\usepackage{amsmath,amsfonts,amsthm} % Math packages
\usepackage{graphicx}
\usepackage{lipsum} % Used for inserting dummy 'Lorem ipsum' text into the template
\usepackage{float}
\usepackage{sectsty} % Allows customizing section commands
\allsectionsfont{\normalfont\scshape} % Make all sections centered, the default font and small caps
\usepackage{url}

\usepackage{fancyhdr} % Custom headers and footers


\usepackage{listings}
\usepackage{xcolor}

\colorlet{punct}{red!60!black}
\definecolor{background}{HTML}{EEEEEE}
\definecolor{delim}{RGB}{20,105,176}
\colorlet{numb}{magenta!60!black}

\lstdefinelanguage{json}{
    basicstyle=\normalfont\ttfamily,
    numbers=left,
    numberstyle=\scriptsize,
    stepnumber=1,
    numbersep=8pt,
    showstringspaces=false,
    breaklines=true,
    frame=lines,
    backgroundcolor=\color{background},
    literate=
     *{0}{{{\color{numb}0}}}{1}
      {1}{{{\color{numb}1}}}{1}
      {2}{{{\color{numb}2}}}{1}
      {3}{{{\color{numb}3}}}{1}
      {4}{{{\color{numb}4}}}{1}
      {5}{{{\color{numb}5}}}{1}
      {6}{{{\color{numb}6}}}{1}
      {7}{{{\color{numb}7}}}{1}
      {8}{{{\color{numb}8}}}{1}
      {9}{{{\color{numb}9}}}{1}
      {:}{{{\color{punct}{:}}}}{1}
      {,}{{{\color{punct}{,}}}}{1}
      {\{}{{{\color{delim}{\{}}}}{1}
      {\}}{{{\color{delim}{\}}}}}{1}
      {[}{{{\color{delim}{[}}}}{1}
      {]}{{{\color{delim}{]}}}}{1},
}


\lstdefinelanguage{python}{
    basicstyle=\normalfont\ttfamily,
    numbers=left,
    numberstyle=\scriptsize,
    stepnumber=1,
    numbersep=8pt,
    showstringspaces=false,
    breaklines=true,
    frame=lines,
    backgroundcolor=\color{background},
    literate=
     *{0}{{{\color{numb}0}}}{1}
      {1}{{{\color{numb}1}}}{1}
      {2}{{{\color{numb}2}}}{1}
      {3}{{{\color{numb}3}}}{1}
      {4}{{{\color{numb}4}}}{1}
      {5}{{{\color{numb}5}}}{1}
      {6}{{{\color{numb}6}}}{1}
      {7}{{{\color{numb}7}}}{1}
      {8}{{{\color{numb}8}}}{1}
      {9}{{{\color{numb}9}}}{1}
      {:}{{{\color{punct}{:}}}}{1}
      {,}{{{\color{punct}{,}}}}{1}
      {\{}{{{\color{delim}{\{}}}}{1}
      {\}}{{{\color{delim}{\}}}}}{1}
      {[}{{{\color{delim}{[}}}}{1}
      {]}{{{\color{delim}{]}}}}{1},
}



\pagestyle{fancyplain} % Makes all pages in the document conform to the custom headers and footers
\fancyhead{} % No page header - if you want one, create it in the same way as the footers below
\fancyfoot[L]{} % Empty left footer
\fancyfoot[C]{} % Empty center footer
\fancyfoot[R]{\thepage} % Page numbering for right footer
\renewcommand{\headrulewidth}{0pt} % Remove header underlines
\renewcommand{\footrulewidth}{0pt} % Remove footer underlines
\setlength{\headheight}{13.6pt} % Customize the height of the header
\usepackage{authblk}
\usepackage[margin=0.89in]{geometry}
\numberwithin{equation}{section} % Number equations within sections (i.e. 1.1, 1.2, 2.1, 2.2 instead of 1, 2, 3, 4)
\numberwithin{figure}{section} % Number figures within sections (i.e. 1.1, 1.2, 2.1, 2.2 instead of 1, 2, 3, 4)
\numberwithin{table}{section} % Number tables within sections (i.e. 1.1, 1.2, 2.1, 2.2 instead of 1, 2, 3, 4)

\setlength\parindent{0pt} % Removes all indentation from paragraphs - comment this line for an assignment with lots of text

%----------------------------------------------------------------------------------------
%	TITLE SECTION
%----------------------------------------------------------------------------------------

\newcommand{\horrule}[1]{\rule{\linewidth}{#1}} % Create horizontal rule command with 1 argument of height

\title{	
\normalfont \normalsize 
\textsc{The Ohio State University, Department of Computer Science and Engineering} \\ [25pt] % Your university, school and/or department name(s)
\horrule{0.5pt} \\[0.4cm] % Thin top horizontal rule
\huge Homework 6: Prediction of rating from Yelp review text % The assignment title
\horrule{2pt} \\[0.5cm] % Thick bottom horizontal rule
}

\author{Grover, Karan \& Arora, Pragya \& Ghai, Piyush}
\affil{\textit {\{grover.120, arora.170, ghai.8\}@osu.edu}}

\date{\normalsize\today} % Today's date or a custom date

\begin{document}
\maketitle % Print the title
\newpage
%----------------------------------------------------------------------------------------
%	PROBLEM 1
%----------------------------------------------------------------------------------------
\section{Problem Statement}
\subsection{About Yelp}
\textbf{Yelp} \cite{yelp} a famous website as well as a mobile app which publishes crowd sourced reviews about food joints and businesses. It also has a division which handles online reservations for restaurants. \textbf{Yelp Dataset Challenge}\cite{yelp_dataset_challenge} is a publicly open contest sponsored by Yelp, in which the participants are challenged to use Yelp's data in an innovative way. \\
\subsection{Our Mission}
The Yelp dataset downloaded from Yelp dataset challenge website is huge and consists of over 2.7M reviews by roughly 687k users for over 86k businesses \cite{yelp_dataset_challenge}. For this project, we chose to predict a review's rating based on the review text. The rating will be done on a scale of 1-5, where 1 stands for awful and 5 stands for excellent. We built multiple models and accessed which models would fit our use case the best. This is explained in more depth in the later sections of this report.

\subsection{About the dataset}
The Yelp Dataset consists of several files in JSON format of the data. The main files as per our use-case were the \textit{yelp\_academic\_dataset\_business.json} \& \textit{yelp\_academic\_dataset\_review.json}. The two data files were \textbf{2.13 GB} \& \textbf{73.6MB} in size respectively. The data representation in the for acadmic\_dataset\_business is as follows : \\
\begin{lstlisting}[language=json]
{
    "type": "business",
    "business_id": (encrypted business id),
    "name": (business name),
    "neighborhoods": [(hood names)],
    "full_address": (localized address),
    "city": (city),
    "state": (state),
    "latitude": latitude,
    "longitude": longitude,
    "stars": (star rating, rounded to half-stars),
    "review_count": review count,
    "categories": [(localized category names)]
    "open": True / False (corresponds to closed, not business hours),
    "hours": {
        (day_of_week): {
            "open": (HH:MM),
            "close": (HH:MM)
        },
        ...
    },
    "attributes": {
        (attribute_name): (attribute_value),
    },
}
\end{lstlisting}
The data representation in the for acadmic\_dataset\_review is as follows : \\
\begin{lstlisting}[language=json]
{   "type": "review",
    "business_id": (encrypted business id),
    "user_id": (encrypted user id),
    "stars": (star rating, rounded to half-stars),
    "text": (review text),
    "date": (date, formatted like "2012-03-14"),
    "votes": {(vote type): (count)},
}
\end{lstlisting}
The given datasets were first converted and exported into csv formats using a simple python script. The python script is a part of the \textbf{PreProcessing1.py} python file.

%% TODO Insert more sections here

\section{Program Description}
The code has been broken down into the following parts:

\begin{enumerate}
\item \textbf{Pre-Processing 1}\\
In pre-processing 1, Yelp provided JSON files were converted into the CSV format. The required CSV files were loaded as Python data-frames and then merged, filtered and stored locally for later usage.

\item \textbf{Pre-Processing 2}\\
In pre-processing 2, the review text is cleaned up, grouped and stored according to the rating of the reviews, which further gets partitioned into training and testing dataset.

\item	\textbf{Models}\\
We implemented a total of 8 models and each model has been separated into a different python file. Following are the models which were implemented and compared:
\begin{itemize}
\item \textbf{Baseline model}
\item \textbf{TFIDF model}
\item \textbf{Bag of words model}
\item \textbf{Bigram model}
\item \textbf{Trigram model}
\item \textbf{Bi and Tri-gram model}
\item \textbf{LDA Model}
\item \textbf{LDA + Sentiment layer model}
\end{itemize}
\end{enumerate}

\section{Coding Contribution}
\subsection{Data Transformation \ Pre-Processing}
Since the files were in JSON format, they were first converted to CSV format. After converting to CSV format. After converting to CSV format, the business and review data frames were joined on \textit{business\_id} for restaurant establishments. From the combined dataset, we got over 1.6M reviews for restaurants alone. The dataset now consisted of just two columns : \textit{Review Text} \& \textit{Star Rating}. \\
\begin{lstlisting}[language=python]
{
  	len(resto_review_data)
  	1630712
}
\end{lstlisting}
The length of reviews was tweaked based on our observations from Exploratory Dataset Analysis. Since a very small review might not contribute meaningfully to our models, we decided to limit the minimum characters in a review to 50, while the maximum character limit was fixed to 500 (this was kept low, because at higher ranges, the dataset was still too huge and required more computing power to run than was possible on our MacBooks).
This further reduced the number of records to a little over 900k. \\
\begin{lstlisting}[language=python]
{
  	resto_review = reduceReviewBasedOnLength(resto_review_data= resto_review_data, minReviewLen=50, maxReviewLen=500)
len(resto_review)
  	910340
}
\end{lstlisting}
\subsection{Cleaning}
Since we are dealing with real world text reviews, the data available to us will contain plenty of punctation words as well common English stopwords. We followed a standard process for text mining to further cleanse and process the review dataset. In this, we first converted all the reviews to \textbf{lowercase}. We then removed \textbf{numbers} \& \textbf{punctuations}. After this, we \textbf{tokenized} the text using the NLTK library's tokenizer. Post this, we removed the \textbf{stopwords} and we also performed \textbf{stemming} of the words. The cleaned dataset was thus used as an input to all our models trained.
\subsection{Creating Training and Testing Corpus}
We used \textbf{70-30} sampling to create a training and testing corpus from the cleaned text reviews. In order to ensure sufficient representation of all the star label values in our training corpus, we split the \textbf{original corpus on the basis of the star labels}, thus creating 5 corpora, one for each star label. The 70-30 sampling was then carried out on each of the 5 corpora and the training and test files thus created were then combined to create one big training and testing corpus. The number of reviews in training corpus for each star label is given  in Table \ref{corpus_size}.
\begin{table}[!htb]
 \centering
 \caption{Star rating distribution in the training corpora}
 \label{corpus_size}
 \begin{tabular}{l l l} 
    \noalign{\smallskip}\hline\noalign{\smallskip}
    Dataset & Count \\
    \noalign{\smallskip}\hline\noalign{\smallskip}
    Rating 1 &58739\\
    Rating 2 &47154\\
    Rating 3 &72107\\
    Rating 4 &174334\\
    Rating 5 &284902\\
    \noalign{\smallskip}\hline
  \end{tabular} 
\end{table}  
\subsection{LDA Model Development}
The LDA\cite{lda} model is present in \textbf{gensim package} in python. The inbuilt library method was not so straightforward and required a \textbf{vectorized bag of words} corpus as an input. It also required a dictionary developed from the available training corpora. The parameters that could be tweaked while developing the model were the corpora size and the total number of topics we want to extract. We chose the \textbf{total topics as 7}. In a normal vectorized corpus, the dimensionality would have been the entire size of the disctionary, which is very huge. Selecting the total topics essentially will reduce the dimensionality of our training corpora to merely 7 selected topics. The topic probability distribution dataset was used as a feature to create new training corpora which was used to train off the shelf classifiers such as \textit{MultinomialNaiveBayes, LogisticRegression, RandomForestClassifier, AdaBoostClassifier}. The performance of the models is discussed in a separate section on Model Evaluation.

\subsection{Contributions from team members}
All team members had equal contributions to this assignment. (33\% each). The key contributions are as follows : 
\begin{enumerate}
   \item Karan Grover.
   \begin{itemize}
     \item Performed the conversion of json to csv.
     \item Suggested the merger of business and reviews file into one.
     \item Performed the cleaning and pre-processing of the dataset.
   \end{itemize}
   \item Pragya Arora
   \begin{itemize}
   \item Performed the exploratory analysis of the dataset, and suggested using a limit size range of reviews.
   \item Suggested and worked on Bi-Grams, Tri Grams \& Bag of Words approach.
   \item Jointly contributed with Piyush in TF-IDF Vectorized model.
   \end{itemize}
   \item Piyush Ghai
   \begin{itemize}
   \item Suggested to keep the representation of all star labels same in the training and testing split.
   \item Suggested and worked on LDA \& LDA + Sentiment Analysis approaches for classification.
   \end{itemize}
\end{enumerate}

\section{Tools and Technology Stacks}
For this assignment, we've used a variety of tools. The coding for this assignment was done using Python. The following is an exhaustive list of modules/tools used : 
\begin{itemize}
   \item Python 2.7
   \item Pandas (For manipulating the dataframes)
   \item nltk (For corpus, dictionary, stopwords, stemming etc)
   \item NumPy (For classification model results)
   \item seaborn (For graphs)
   \item cPickle (For saving pickle files and using them later in models)
   \item iPython (The interactive python shell, which was used for development)
   \item sci-kit learn (For various classification algorithms)
   \item Gensim (For LDA model)
   \end{itemize}


\newpage
\begin{thebibliography}{}
 \bibitem{yelp}
 ggplot Library is used in this assignment to plot most of the graphs in this assgnment \url{http://ggplot2.org/}. 
 
 \bibitem{yelp_dataset_challenge}
 ggplot Library is used in this assignment to plot most of the graphs in this assgnment \url{http://ggplot2.org/}. 
 
 \bibitem{nltk}
 ggplot Library is used in this assignment to plot most of the graphs in this assgnment \url{http://ggplot2.org/}. 
 
 \bibitem{lda}
 ggplot Library is used in this assignment to plot most of the graphs in this assgnment \url{http://ggplot2.org/}. 
  
\end{thebibliography}


\end{document}